\section{Leer Aktiwiteite}
    \subsection{Kontaktyd en Leerure}
        \begin{minipage}{0.4\linewidth}
            Aantal lesings per week: \\
            Tutoriaalsessies per week:
        \end{minipage}
        \begin{minipage}{0.4\linewidth}
            4 lesings, 50 minute per lesing \\
            1 sessie, 100 minute per sessie
        \end{minipage}

        Hierdie module dra 'n gewig van 18 krediete, wat aandui dat 'n student 
        gemiddeld 180 ure moet bestee om die vereiste vaardighede te bemeester
        (insluitend die tyd wat spandeer word vir die voorbereiding van toetse
        en eksamens).  Dit beteken dat die student gemiddeld 12 ure per week se studietyd
        moet afstaan aan hierdie module.  Die geskeduleerde kontaktyd vir hierdie
        module is ongeveer 7 ure, wat beteken dat nog 5 ure aan selfstudie moet toegewy word.
    
    \subsection{Lesings}
        \begin{table}[!h]
            \begin{center}
             \begin{tabular}{|l|c|p{2.1cm}|p{2.1cm}|p{2.1cm}|p{2.1cm}|p{1.5cm}|}
                 \hline
                 {\bf \#} & {\bf Tyd} & {\bf Ma.} & {\bf Dins.} & {\bf Woens.} &
                 {\bf Do.} & {\bf Vry.} \\
                 \hline
                 1  & 07:30--08:20 &  &  &  &  & \\ \hline
                 2  & 08:30--09:20 &  Ing III -- 1 (Afr)  &  & Ing III -- 6 (Afr) &  & \\ \hline
                 3  & 09:30--10:20 &  &  &  & Ing III -- 6 (Afr) & \\ \hline
                 4  & 10:30--11:20 &  &  &  &  & \\ \hline
                 5  & 11:30--12:20 &  &  &  & Thuto 1 -- 1 (Eng) & \\ \hline
                 6  & 12:30--13:20 &  &  &  & Thuto 1 -- 1 (Eng) & \\ \hline
                 7  & 13:30--14:20 &  &  &  &  & \\ \hline
                 8  & 14:30--15:20 &  &  &  &  & \\ \hline
                 9  & 15:30--16:20 &  & Thuto 1 -- 2 (Eng) &  &  & \\ \hline
                 10 & 16:30--17:20 &  & Thuto 1 -- 2 (Eng) &  &  & \\ \hline
                 11 & 17:30--18:20 &  & Ing III -- 6 (Afr) &  &  & \\
                 \hline
             \end{tabular}
             \caption{Lesings rooster}
            \end{center}
        \end{table}
        
	Lesing bywoning en deelname aan besprekings is verpligtend.
    Aangesien die inhoud van elke lesing volg op di\'{e} van die vorige lesings,
    is dit in die student se eie belang om die materiaal te bestudeer 
    en nie 'n lesing te mis nie. As 'n student 'n  
    lesing nie kan bywoon nie, is dit die verantwoordelikheid 
    van die student om die studiemateriaal te kry en die werk in te haal. 
    Geen individuele lesings sal aangebied word nie. Enige vorige materiaal sal selde 
    herhaal word in 'n volgende lesing.
        
	Daar word van studente verwag om voor te berei vir die lesings. Aangesien
	'n groot volume van werk gedek moet word, is dit nie moontlik om elke
	aspek van die werk te bespreek nie. Studente moet dus die studienotas 
	deeglik deurlees en vooraf weet wat in die volgende lesing gedek gaan word, om 
	hulle in staat te stel om enige iets wat onduidelik is maklik te identifiseer.
        
	Dr Wilke sal die Afrikaans lesings aanbied en Mnr Page sal die Engelse lesings
	aanbied (volgens die rooster op die vorige bladsy).  Die lesings is verdeel
	soos aangedui op die rooster.  Verwys asseblief na Click-UP vir enige
	veranderinge in verband met die rooster.
        
    \subsection{Huiswerk Probleme}
	Die dosent mag analitiese probleme gee om die student se onafhanklike
	vaardighede met probleemformuleering en analiseering te verbeter.
	Studente moet die huiswerkprobleme (tutoriaalprobleme)
	op hulle eie tyd oplos.  Die probleme hoef nie ingehandig te word nie
	en sal ook nie ge\"{e}valueer word nie.  Die oplossings van hierdie 
	probleme sal nie beskikbaar gemaak word nie.
	
	Dit sal vir die student voordelig wees om hierdie probleme te voltooi
	aangesien soortgelyke vorme van die probleme in die toets of eksamens
	gevra kan word.  Die student mag gebruik maak van die besprekingsbord op Click-UP
	as hy/sy spesifieke vrae het oor 'n gegewe huiswerk probleem.
    
    \subsection{Tutoriale}
        \begin{table}[!h]
            \begin{center}
             \begin{tabular}{|l|l|l|l|}
                 \hline
                 {\bf Dissipline} & {\bf Dag} & {\bf Tyd} & {\bf Plek} \\
                 \hline
                 b3 B2     & Dond. & 14:30-16:20 & NWII Lab 1,2 \\
                 c3s3 C2   & Ma.   & 15:30-17:20 & NWII Lab 2,3,4 \\
                 m3 M2     & Vry.   & 13:30-15:20 & NWII Lab 1,2 \\
                 n3p3 N2P2 & Woens.   & 11:30-13:20 & NWII Lab 1,2 \\
                 M2        & Dins.  & 07:30-09:20 & NWII Lab 2,3,4 \\
                 S3        & Ma.   & 11:30-13:20 & NWII Lab 1,2,3,4 \\
                 \hline
             \end{tabular}
             \caption{Tutoriaal rooster}
            \end{center}
        \end{table}
        
	Tutoriaal sessies sal gebruik word vir verdere ontwikkeling van 
	die studente se begrip en kennis oor die lesingsonderwerpe. 
	Die tutoriaalsessies dien as hulpmiddel om die student in staat te stel om verskeie
	ingenieurs- en wiskundigeprobleme op te los. 

	Tutoriaalsessies se bywoning en deelname is verpligtend.
	Aangesien die inhoud van elke tutoriaalsessie volg op di\'{e} van die
	vorige tutoriaalsessie, is dit in die student se eie belang om die lesingsmateriaal
	gereeld te hersien en nie 'n tutoriaalsessie te mis nie.
	

