    \subsection{Departementele Studiegids}
    \label{sec:department}
    Die studiegids is 'n belangrike deel van die algemene studiegids
    van die Departement. In die Departementele studiegids word
    informasie gegee oor die visie en missie van die departement,
    algemene administrasie en regulasies (professionaliteit en
    integriteit, kursus verwante inligting en formele kommunikasie,
    werkswinkel gebruik en veiligheid, plagiaat, klasverteenwoordiger
    pligte, sieketoets en siek eksamenriglyne, vakansiewerk,
    universiteitsregulasies, vrae wat dikwels gevra word), ECSA
    uitkomste en ECSA uittreevlak uitkomste, ECSA kennisarea, CDIO,
    nuwe kurrikulum en assessering van kognitiewe vlakke.  Daar word
    verwag dat jy baie vetroud word met die inhoud van die
    Departementele Studiegids.  Dit is beskikbaar in 
\href{http://web.up.ac.za/sitefiles/file/44/1026/2163/noticeboard/%
        DepartmentalStudyGuide\_Eng\_2014.pdf}{English}
    en
    \href{http://web.up.ac.za/sitefiles/file/44/1026/2163/noticeboard/%
        Departementele\_Studiegids\_Afr\_2014(1).pdf}{Afrikaans}
    op die
    \href{http://web.up.ac.za/default.asp?ipkCategoryID=11426&%
        subid=11426&ipklookid=7}{Departementele webwerf.}

    \noindent
    \textbf{English:} \\
    \url{http://web.up.ac.za/sitefiles/file/44/1026/2163/noticeboard/%
      DepartmentalStudyGuide\_Eng\_2014.pdf} \\
    \textbf{Afrikaans:} \\
    \url{http://web.up.ac.za/sitefiles/file/44/1026/2163/noticeboard/%
      Departementele\_Studiegids\_Afr\_2014(1).pdf} \\~\\

    \noindent
    \textbf{Neem kennis van die \uline{spesifieke instruksies} soos
      uiteengesit in die studiegids hierbo:}
    \begin{itemize}
        \item \textbf{Veiligheid}
        \item \textbf{Plagiaat}
        \item \textbf{Wat om te doen indien jy siek was (baie belangrik)?}
        \item \textbf{App\'el proses vir punte aanpassings}
    \end{itemize}