\section{Assesserings Proses}
    Verwys na die eksamen regulasies in die Jaarboek van die Fakulteit
    Ingenieurswese, Bou-Omgewing en IT.

    Om die module deur te kom moet die student:
    \begin{itemize}
        \item 'n Finale punt van 50\% behaal \\ {\bf en}
        \item 'n Sub-minimum van 40\% vir die semester punt \\ {\bf en}
        \item 'n Sub-minimum van 40\% vir die finale eksamen
%         \item Obtain a subminimum of 50\% for the ECSA exit outcome 5 
%               assessment matrix for semester test 1, semester test 2, 
%               sick test (if applicable), exam and re-exam (if applicable).
    \end{itemize}
    
    \subsection{Berekening van die Finale Punt}
        Die finale punt word as volg bereken:
        \begin{itemize}
            \item Semester punt: 50\%
            \item Finale eksamen: 50\% (3-uur eksamen), toe-boek
        \end{itemize}

    \subsection{Berekening van die Semester Punt}
        Besonderhede rakende die berekening van die semester punt word in die volgende tabel gelys:
        \begin{table}[!h]
            \begin{center}
             \begin{tabular}{|p{5cm}|c|l|l|}
                 \hline
                 {\bf Evaluasie Metode} & {\bf Aantal} & 
                 {\bf Bydrae van elkeen} & {\bf Totaal} \\
                 \hline
                 Semester toets, geskryf, toe-boek 
                    & 2 & 40\% & {\bf 80\%} \\ \hline
                 Semester Projek 
                    & 1 & 20\% & {\bf 20\%} \\
                 \hline
                 \multicolumn{3}{|l|}{{\bf Total}} & {\bf 100\%} \\
                 \hline
             \end{tabular}
             \caption{Berekening van Semester Punt}
            \end{center}
        \end{table}
    
    \subsection{Semester Toetse}
	Twee semester toetse sal geskry word gedurende die semester, in die 
	week van 9 tot 16 Maart 2013 en 4 tot 11 Mei 2013. Semester toetse is 90 minute lank.
	Die sillabus wat in die toets gedek word
	sal die week voor toetsweek bespreek word.  Altwee toetse sal 
	toe-boek wees.
        
	Die memorandums van die geskeduleerde toetse sal in elektroniese formaat
	op Click-UP beskibaar wees na die onderskeie toetse. Die student is welkom om gebruik te maak van die 
	besprekingsbord vir enige verdere vrae oor die toetse.
	
    \subsection{App\`{e}lle en navrae oor punte}
	Die punte toegeken vir die werksopdragte en semester toetse sal 
	beskikbaar gemaak word op Click-UP. Indien die student enige navrae het oor die toegekende punt
	moet die student asseblief binne 14 dae vanaf 
	die punte ontvang is die prodedure onder volg. 
	Na die verloop van 14 dae sal geen veranderings 
	aangebring word nie.
	
	Die dosent sal wag tot al die vraestelle ontvang is en tot die 14 dae verby is voor die 
	hersieningsproses aangepak word. 

        \subsubsection{App\`{e}l Proses}
            \begin{enumerate}
                \item Laai die memorandum af van Click-UP en merk die relevante
                    vrae met 'n potlood.
                \item Tel die punte op vir die vraag.
                \item Skryf die vraag nommer aan die voorkant van jou vraestel 
                    neer.
                \item Handig jou vraestel aan die dosent teen die einde van die 
                    lesing.
            \end{enumerate}

            Indien die punt verander, sal die dosent dit korrigeer en die vraestel teken.
        
        \subsubsection{Afwesigheid van die Toets(e) en/ of Eksamen}
            Verwys na die Departmentele studiegids.
        
        \subsubsection{Plagiaat}
            Verwys na die Departementele studiegids.



