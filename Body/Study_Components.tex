\section{Study Components}
        \begin{table}[!h]
             \begin{tabular}{|p{1.4cm}|l|p{2cm}|p{2cm}|}
                 \hline
                 {\bf Theme No.} & {\bf Topic} & 
                    {\bf Notational Hours} & {\bf Contact Sessions} \\
                 \hline
                 1  & Introduction to Computers and Programming &    &   \\
                 \hline
                 2  & Basic Programming                         & 12 & 3 \\
                    & \qquad Using Python as a Calculator       &    &   \\
                    & \qquad Names and Objects                  &    &   \\
                 \hline
                 3  & Control Statements                        & 24 & 6 \\
                    & \qquad Looping                            &    &   \\
                    & \qquad Branching                          &    &   \\                 
                 \hline
                 4  & Data Containers                           & 28 & 7 \\
                    & \qquad Lists, tuples, dictionaries        &    &   \\
                    & \qquad Vectors and Matrices               &    &   \\                 
                 \hline
                 5  & Structured Programming                    & 24 & 6 \\
                    & \qquad User-defined Functions             &    &   \\                 
                    & \qquad Code Structure                     &    &   \\
                    & \qquad Local Variables                    &    &   \\
                 \hline
                 6  & Plotting and Graphs                       & 20 & 4 \\
                    & \qquad 2D Graphs                          &    &   \\
                    & \qquad Graph Annotation                   &    &   \\
                    & \qquad 3D Graphs                          &    &   \\
                 \hline
                 7  & File Handling                             & 8  & 2 \\
                    & \qquad Reading and Writing to Files       &    &   \\                 
                    & \qquad Directory Management               &    &   \\
                 \hline
                 8  & High Level Programming                    & 12 & 3 \\
                    & \qquad Error Handling                     &    &   \\
                    & \qquad Additional Modules                 &    &   \\    
                    & \qquad Advanced Programming               &    &   \\
                 \hline
                 9  & Spreadsheets                              & 28 & 7 \\
                    & \qquad Formulas and Calculations          &    &   \\
                    & \qquad Spreadsheet Detective              &    &   \\
                    & \qquad Plotting and Graphs                &    &   \\
                    & \qquad Linear Programming                 &    &   \\
                    & \qquad Non-Linear Solver problems         &    &   \\
                    & \qquad Data Pilot                         &    &   \\
                    & \qquad What-if Scenarios                  &    &   \\
                    & \qquad Visual Formatting                  &    &   \\
                 \hline
                 10 & HTML, Websites and the Internet           & 16 & 4 \\
                 \hline
                 11 & Databases                                 & 8  & 2 \\
                 \hline
                    & {\bf TOTAL}                               &180 &44 \\
                 \hline
             \end{tabular}
             \caption{Module Structure}
        \end{table}
        
    \subsection{Purpose of the module}
        {\bf Advanced spreadsheet applications:} \\
        Named ranges, linear algebra,
        solution of systems of equations, regression, interpolation,
        optimization and table manipulation. \\ \\
        {\bf Basic structured programming:} \\
        Looping, branching, subroutines, iteration, reading and writing data files.
        Development, coding and debugging of simple programs in a high level 
        programming language. Programming principles are illustrated via 
        mathematical concepts such as limits, differentiation, integration
        and linear algebra. Structured programming by making use of functions
        and available packages. Basic graphical output (plotting) is also
        covered.
    
    \subsection{Module Structure}
        The structure of this module is shown in the table above. The total module 
        hours (notational hours) include the contact time, as well as the estimated time to be 
        allocated for self-study, preparation of assignments,
        tests and the examination. The mode of instruction is via lectures,
        tutorial class and assignment.

    \subsection{Lecture Plan}
        \begin{table}[!h]
            \begin{center}
             \begin{tabular}{|l|l|l|l|l|}
                 \hline
                 {\bf Week} & {\bf Lectures} & {\bf Dates} & {\bf Study Theme} & {\bf Tutorial} \\
                 \hline
                 1  & 4     & 11 –- 15 Feb     & 1, 2  & Special \\
                 2  & 4     & 18 –- 22 Feb     & 3     & T1 \\
                 3  & 4     & 25 Feb –- 1 Mar  & 3, 4  & T2 \\
                 4  & 4     & 4 –- 8 Mar       & 4     & T3 \\
                    &       & == TESTWEEK 1 == &       & \\
                 5  & 2     & 18 -- 19 Mar     & 4     & T4 \\
                    &       & ==== RECESS ==== &       & \\
                 6  & 3 / 4 & 2 -- 5 Apr       & 5     & T4 \\
                 7  & 4     & 8 -- 12 Apr      & 5, 6  & T5 \\
                 8  & 4     & 15 -- 19 Apr     & 6     & T6 \\
                 9  & 4     & 22 -- 26 Apr     & 9     & T7 \\
                 10 & 3 / 2 & 29 Mar -- 3 May  & 9     & T8 \\
                    &       & == TESTWEEK 2 == &       & \\
                 11 & 4     & 13 -- 17 May     & 7, 11 & T9 \\
                 12 & 4     & 20 -- 24 May     & 10    & T10 \\
                 13 & 4     & 27 -- 30 May     & 8     & T11 \\
                    &       & ===== EXAM ===== &       & \\
                 \hline
             \end{tabular}
             \caption{Lecture Plan}
            \end{center}
        \end{table}
        
        This outlines the lecture plan for the semester. Students should use
        this plan to make sure that they do not fall behind the class
        lectures and/or tutorials.

        The table below lists which study themes will be covered in which weeks
        of the semester. Each week has a corresponding tutorial which should
        be completed during the specific week in the tutorial sessions.
        
        Selected exercise problems, from the study notes, will be covered during 
        the tutorial sessions. The selected chapters and problems (from the study
        notes) that will be covered in each tutorial session will be added to 
        Click-UP.
        
        NOTE: The ``Special'' tutorial session in week 1 will only be for students
        who are re-taking this course. It is imperative for students (who are 
        re-taking this course) to attend the first lectures of the semester and to 
        attend the ``Special'' tutorial session. The programming language used
        last year (Octave) will be changed to Python this year and the ``Special''
        tutorial session will be used to show the differences between the two
        and highlight the fundamental difference in behaviour of Python.
        
    \subsection{Notes}
        Notational hours include contact time, as well as the estimated time 
        necessary for preparation for tests and exams. Contact sessions indicate
        the regular lectures. The number of contact sessions per chapter is 
        tentative. It may change depending on the progress during lectures.
        
        Please note that some subsections in a chapter of the textbook will not be
        covered during the lectures, whereas other subsections may be given as 
        self-study. The lecturer will provide information about the sub-sections
        that will not be covered for test- and exam purposes.
    
    
    \subsection{Study Theme Descriptions}
        \subsubsection{Theme 1: Introduction to Computers and Programming}
            The introduction allows the student to obtain a general overview. 
            This section will not be explicitly tested. It is important to note
            that an understanding of computer architecture and flow diagrams
            will make the following sections more accessible.

            
        \subsubsection{Theme 2: Basic Programming}
            \paragraph{Using Python as a Calculator:}
                The student has to ensure that he/she is comfortable with the 
                \emph{IPython Console} and \emph{Spyder} environments. It is the
                student’s responsibility to spend enough time using these 
                environments throughout the semester.

            \paragraph{Names and Objects:}
                The student has to be familiar with the guidelines in the class 
                notes on names and objects e.g. objects are created in memory and
                a name in bound to that object. The student must also be familiar
                with the memory model of Python. The student must also know the
                difference between the object types e.g. int, float, str, bool.

                
        \subsubsection{Theme 3: Control Statements}
            \paragraph{Looping:}
                The student has to be able to identify the usage of unconditional 
                (for) and conditional (while) loops from a problem statement. 
                The student has to be familiar with the Python's syntax and 
                indentation and be able to implement conditional and unconditional
                loops in a program to perform a given task.
            
            \paragraph{Branching:}
                The student has to be able to identify the usage of branches from 
                a problem statement. The student has to be familiar with 
                Python's syntax and indentation and be able to implement branches 
                in a program to perform a given task.

            
        \subsubsection{Theme 4: Data Containers}
            \paragraph{Lists, tuples and dictionaries:}
                The student has to understand the principles of lists, tuples and 
                dictionaries as well as the Python syntax thereof. The student must
                be able to identify and be able to implement lists, tuples and 
                dictionaries. The student must be able to rewrite an existing 
                computer program that uses variables such that it uses either
                lists, tuples or dictionaries.
                
            \paragraph{Vectors and Matrices:}
                The student has to understand the principles of vectors and 
                matrices as well as the Python syntax thereof. The student must
                be able to identify and be able to implement vectors and matrices.
                The student must understand the basic vector and matrix operations
                and be able to implement it in a program. The student must be able
                to rewrite an existing computer program that uses variables
                such that it uses vectors and matrices.

        
        \subsubsection{Theme 5: Structured Programming}
            The student must be familiar with functions, subfunctions and code 
            structure. The student must be able to create functions with multiple 
            inputs and outputs as well as optional (keyword) inputs. The student 
            must be able to rewrite an existing computer program such that it makes
            use of functions and subfunctions. The student has to be familiar with 
            local variables and understand their scope within a function, 
            subfunction or module.

            
        \subsubsection{Theme 6: Plotting and Graphs}
            The student must be able to create 2D plots i.e. the student must be 
            familiar with different presentations of data, presenting multiple 
            graphs on the same figure, displaying grids, labelling of the axis, 
            naming of figures, using legends and scaling axis systems.
            The student must be able to create 3D figures. The student has to be 
            familiar with creating surface and mesh plots.

            
        \subsubsection{Theme 7: File Handling}
            The student must be familiar with reading and writing to and from 
            files, as well as the various file types in Python. The student must 
            be able to store the data in a given format. The student must also
            understand how to access different directories from within the Python
            program.

        
        \subsubsection{Theme 8: High Level Programming}
%             \paragraph{Error Handling:}
%                 
%                 
%             \paragraph{Advanced Programming:}
            
%             \paragraph{Additional Modules:}
                The student must be able to solve over and under determined linear
                systems of equations, non-linear systems of equations and systems 
                of linear differential equations using additional modules and 
                functions in Python. The student must be able to manipulate 
                polynomials, conduct optimization, integrate numerically, obtain
                descriptive statistical measures of data and interpolate numerical 
                data using using additional modules and functions in Python.
                
                The student must be able to navigate Python's documentation 
                as well as online documentation and be able to find, read and 
                understand the documentation of the required functions in order 
                to solve specific problems. The student should therefore be able
                to independently increase his/her knowledge of the Python
                programming language.

        \subsubsection{Theme 9: Spreadsheets}
            \paragraph{Formulas and Calculations:}
                The student must be able to solve problems using formulas. The
                student must be able to solve problems using functions that he/she
                are familiar with, as well as use the function wizard for unknown
                functions. The student must be able to name ranges of cells and use
                them in calculations.

                The student must be able to use filters to filter data and perform
                calculations on the filtered dataset. The student must be able to
                perform calculations on data that is split over multiple
                spreadsheets.

            \paragraph{Spreadsheet Detective:}
                The student must be able to use the spreadsheet detective to
                familiarise themselves with the dependencies in an unknown
                spreadsheet. The student must be able to switch between formula
                and value view.

            \paragraph{Plotting and Graphs:}
                The student must be able to create and customise charts. The
                student must be able to find trends in data.

            \paragraph{Linear Programming:}
                The student must be able to solve linear programming problems
                using LibreOffice Calc. The student must be able to solve
                problems with mixed integer, real and boolean variables. The
                student must be able to accommodate multiple constraints.

            \paragraph{Non-Linear Solver:}
                The student must be able to solve non-linear one dimensional
                problems using goal seek in LibreOffice Calc.

            \paragraph{Data Pilot:}
                The student must be able to extract relational data using the Data
                Pilot in LibreOffice Calc.

            \paragraph{What-if Scenarios:}
                The student must be able to perform what-if investigations using
                the scenario function in LibreOffice Calc.

            \paragraph{Visual Formatting:}
                The student must be able to improve the appearance of a worksheet
                by using colours, highlighting, borders and font properties. The
                student must be able to perform conditional formatting of cells.
        
            
        \subsubsection{Theme 10: HTML, Websites and the Internet}
            The student should understand that basic websites are just a 
            text file with special HTML tags. The student should be able to 
            modify existing html websites. Finally the student should know 
            how to do advanced searches on the internet for information.
        

        \subsubsection{Theme 11: Databases}
            The student should know and understand what a database is and when a 
            database is the correct data structure to store data. The student 
            should be able to create a database of information using a database 
            software package and should be able to create queries and reports 
            based on the data in the database.