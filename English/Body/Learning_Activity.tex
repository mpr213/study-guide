\section{Learning Activities}
    \subsection{Contact Time and Learning Hours}
        \begin{minipage}{0.4\linewidth}
            Number of lectures a week: \\
            Tutorial sessions a week:
        \end{minipage}
        \begin{minipage}{0.4\linewidth}
            4 lectures, 50 minutes per lecture \\
            1 session, 100 minutes per session
        \end{minipage}

        This module carries a weight of 18 credits, indicating that a
        student should spend an average of 180 hours to master the
        required skills (including time spent preparing for tests and
        examinations). This means that you should devote an average of
        13 hours of study time per week to this module. The scheduled
        contact time is approximately seven hours per week, which
        means that another six hours per week of own study time
        should be devoted to the module.

    \subsection{Lectures}
        \begin{table}[!h]
            \begin{center}
             \begin{tabular}{|l|c|p{1.3cm}|p{2.8cm}|p{2.1cm}|p{2.6cm}|p{1.3cm}|}
                 \hline
                 {\bf \#} & {\bf Time} & {\bf Mon.} & {\bf Tues.} & {\bf Wed.} &
                 {\bf Thurs.} & {\bf Fri.} \\
                 \hline
                 1  & 07:30--08:20 &  & [E2]vd Bijl Hall & [A]Ing III--6 &  & \\ \hline
                 2  & 08:30--09:20 &  & [E2]vd Bijl Hall & [A]Ing III--6 &  & \\ \hline
                 3  & 09:30--10:20 &  &  &  &  & \\ \hline
                 4  & 10:30--11:20 &  &  &  &  & \\ \hline
                 5  & 11:30--12:20 &  &  &  & [E1]Thuto 1--1 & \\ \hline
                 6  & 12:30--13:20 &  &  &  & [E1]Thuto 1--1 & \\ \hline
                 7  & 13:30--14:20 &  &  &  &  & \\ \hline
                 8  & 14:30--15:20 &  &  &  &  & \\ \hline
                 9  & 15:30--16:20 &  & [E1]Thuto 1--2 &  & [E2] Eng III--1 & \\ \hline
                 10 & 16:30--17:20 &  & [E1]Thuto 1--2 \: [A]Ing III--6 &  & [E2] Eng III--1 & \\ \hline
                 11 & 17:30--18:20 &  & [A]Ing III--6 &  &  & \\
                 \hline
             \end{tabular}
             \caption{Lecture Time Table}
            \label{tab:lectures}
            \end{center}
        \end{table}

        Dr Wilke will present the \textbf{Afrikaans} lectures and Mr Page the
        \textbf{English} lectures as shown in Table \ref{tab:lectures}. The
        lectures are split as per the time table above. Please refer to your
        individual timetable as to which English lecture group you should
        attend.  Also refer to {\it ClickUP} for any time table updates.

        Lecture attendance and participation in discussions are compulsory.
        Since the contents of each lecture follow on those of previous
        lectures, it is in the students’ own interest to study the material
        covered on a regular basis and not to miss a lecture. However, should a
        student not be able to attend a certain lecture for whatever reason,
        the onus is on him / her to obtain the study material and catch up on
        the work. No individual lectures will be presented. Previous material
        will rarely be repeated in a following lecture.

        Students are expected to prepare for lectures. Since a large volume of
        work needs to be covered, it is not possible to lecture every aspect in
        the finest detail. Students should therefore read the textbooks
        thoroughly and already know beforehand what the next lecture is about
        in order to identify anything that is unclear. The lecturer may also
        assign some sections of the study notes and lecture notes / handouts
        for self-study.  These sections will be part of the syllabus, but will
        not be discussed in the class.

    \subsection{Tutorials}
        \begin{table}[!h]
            \begin{center}
            \begin{tabular}{|l|l|l|l|}
                \hline
                {\bf Discipline} & {\bf Day} & {\bf Time} & {\bf Venue} \\
                \hline
                S3          & Mon.   & 11:30-13:20 & NWII Lab 1,2,3,4 \\
                c3 s3 C2    & Mon.   & 15:30-17:20 & NWII Lab 2,3,4 \\
                M2          & Tues.  & 07:30-09:20 & NWII Lab 2,3,4 \\
                n3 p3 N2 P2 & Wed.   & 11:30-13:20 & NWII Lab 1,2,3,4 \\
                b3 B2       & Thurs. & 14:30-16:20 & NWII Lab 1,2,3,4 \\
                m3 M2       & Fri.   & 13:30-15:20 & NWII Lab 1,2 \\
                \hline
            \end{tabular}
            \caption{Tutorial Time Table}
            \label{tab:tutorials}
            \end{center}
        \end{table}

        Tutorial sessions will be used to further develop the students
        understanding and knowledge of the topics covered in lectures
        by solving numerous engineering and mathematical problems with
        the tools introduced in this module. The tutorial sessions are
        tabulated in Table \ref{tab:tutorials}, and allocated
        according to discipline\footnote{Discipline Symbols:
            B -- Industrial and Systems;
            C -- Chemical;
            M -- Mechanical and Aeronautical;
            N -- Materials Science and Metallurgical;
            P -- Mining; and
            S -- Civil. \\
            (uppercase letter -- four year plan;
            lowercase letter -- five year plan)}.

        Each week's tutorial session will be used to review the
        previous week's lecture content and concepts. Selected
        exercise problems, from the study notes, will be covered
        during the tutorial sessions. The selected chapters and
        problems, that will be covered in each tutorial session, will
        be added to {\it ClickUP}.

        It is vital that students do these tutorial problems on their
        own, prior to attending the tutorial session, and use the
        tutorial session for getting help with problems and / or
        concepts that the student is struggling with. It is important
        to understand that the tutors are there to assist the student
        to grasp difficult concepts that limits a student to solve a
        problem and not to solve the problems on their behalf.

        It will be to the benefit of the student to complete these
        problems as solving these problems will develop the students
        thinking and programming ability as the tests and exam will
        test the ability of the student to solve new problems of
        similar complexity.

        \textbf{Attendance} of the \textbf{tutorial sessions}
        are \textbf{optional} but \textbf{handing in} of the
        \textbf{tutorial assignments} on {\it ClickUP}
        \textbf{compulsory}, as they will be \textbf{graded and count
        towards the semester mark}.
