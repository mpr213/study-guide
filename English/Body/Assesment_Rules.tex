\section{Rules of Assessment}
    Refer to the exam regulations in the Yearbooks of the Faculty of Engineering,
    Built Environment and Information Technology.

    To pass the subject a student must:
    \begin{itemize}
        \item Obtain a final mark of at least 50\%; {\bf and}
        \item Obtain a subminimum of 40\% for the semester mark; {\bf and}
        \item Obtain a subminimum of 40\% for the final examination
    \end{itemize}

    \subsection{Determination of Final Mark}
        The final mark is compiled as follows:
        \begin{itemize}
            \item Semester mark: 50\%
            \item Final Exam mark: 50\% (3-hour exam), closed-book
        \end{itemize}

    \subsection{Determination of Semester Mark}
        The semester mark will be determined as shown in the table below:
        \begin{table}[!h]
            \begin{center}
             \begin{tabular}{|p{5cm}|c|l|l|}
                 \hline
                 {\bf Evaluation Method} & {\bf No. of} &
                 {\bf Contribution of ea.} & {\bf Total} \\
                 \hline
                 Semester tests (written, closed book)
                    & 2 & 37.5\% & {\bf 75\%} \\ \hline
                 Semester Project
                    & 1 & 15\% & {\bf 15\%} \\ \hline
                 Tutorial Assignments
                    & 5 &  2\% & {\bf 10\%} \\
                 \hline
                 \multicolumn{3}{|l|}{{\bf Total}} & {\bf 100\%} \\
                 \hline
             \end{tabular}
             \caption{Determination of Semester Mark}
            \end{center}
        \end{table}

    \subsection{Semester Tests}
        Two tests will be written during the semester, in the weeks 8 to 15
        March 2014 and 5 to 10 May 2014. The duration of each test will be
        90 minutes. Syllabi of the tests will be announced during the lecture
        week preceding the test week. Both tests will be closed-book.
        Additional test instructions will be announced during the lecture
        week preceding the test week and uploaded to Click-UP.

        Memoranda on the scheduled tests will be made available in electronic
        format on Click-UP and will not be discussed during lectures. You may
        make use of the discussion board on Click-UP should you have a specific
        question regarding a problem given in the semester tests.

    \subsection{Appeals and queries on marks}
        The marks awarded for assignments and semester tests will be
        posted on Click-UP. If you have appeals or queries regarding marks
        you must follow the procedure outlined below within 14 days from
        receiving the marks. After the 14 days no marks will be altered.

        I will wait until I have received all the question papers from
        students and until the 14 days have passed before looking at the
        papers again.

        \subsubsection{Appeal Process}
            \begin{enumerate}
                \item Download the memorandum, feedback and query documents
                    from Click-UP.
                \item Go through your test paper carefully along with the
                    memorandum and feedback documents.
                \item Fill out the query form, describing your appeal or query,
                    and attached it to your question paper.
                \item Hand your question paper in to the lecturer at the end of
                    the lecture within the 14 day window period.
            \end{enumerate}

            I will then just check whether or not you marked your paper
            correctly and sign next to it if it is indeed correct.
            \textbf{No paper will be accepted without a query form attached.}

        \subsubsection{Absence from Test(s) and/or Exams}
            Refer to the Study guide of the Department.

        \subsubsection{Re-Exams / Supplementary Exams}
            Refer to the Study guide of the Department.

        \subsubsection{Plagiarism}
            Refer to the Study guide of the Department.
