\section{Study Material and Software}
    \subsection{Prescribed Textbook}
        There is no prescribed textbook for this course, however the following
        study notes have been developed at UP and will be made available on
        {\it ClickUP}:

        ``\underline{Introduction to programming for engineers using Python}''
        by Logan G. Page, Daniel N. Wilke and Schalk Kok.

    \subsection{Complementary Sources}
        The following complementary source of notes will also be made available
        on {\it ClickUP}:

        ``\underline{Python for Computational Science and Engineering}'' by
        Hans Fangohr

        Numerous other complementary sources for this course are mentioned in
        the study notes above. The discussion forum and wiki page on the
        {\it ClickUP} system will also be utilised for additional explanations and
        examples on various topics covered in this course.

        Any notes on study material not covered in the study notes will be made
        available in electronic format on {\it ClickUP}. These additional notes will
        also be part of the syllabus. Lecture slides will be made available on
        {\it ClickUP}. Please note that these lecture slides do not cover all the
        work discussed in class and students should take down their own
        supplementary notes during lectures.

        Problem solutions covered in detail during the lectures will not be
        made available again at a later stage.

    \subsection{Required Software}
        In this course we make use of open source software. This implies
        that students can freely copy and use the software legally
        without any restrictions.

        The programming package used in this course is Python. Python is
        high level programming language used in many disciplines around the
        world. Python is a well suited programming language for engineers as
        it is easy to learn, well supported and documented, relatively fast,
        and well suited for numerical and scientific computing.

        LibreOffice Calc will be used for spreadsheets in this course. It is
        in many ways very similar to Microsoft Excel with the exception
        that it can be downloaded and used without any restrictions.

        \noindent
        You can download Python ({\tt Python(x,y)-2.7.9.0}) from:
        \begin{itemize}
            \item Windows: \url{ftp://ftp.ee.up.ac.za/pub/windows/python/Python(x,y)-2.7.9.0.exe}
            \item Mac: None (Please install Spyder IDE below)
            \item Vendor: \url{https://python-xy.github.io/downloads.html}
        \end{itemize}
        (See the study notes, mentioned above, for installation instructions)

        \noindent
        You can download LibreOffice
        ({\tt LibreOffice\_5.0.4}) from:
        \begin{itemize}
            \item Windows: \url{ftp://ftp.ee.up.ac.za/pub/windows/libreoffice/LibreOffice_5.0.4_Win_x86.msi}
            \item Mac: \url{ftp://ftp.ee.up.ac.za/pub/mac/libreoffice/LibreOffice_5.0.4_MacOS_x86-64.dmg}
            \item Vendor: \url{http://www.libreoffice.org/download/libreoffice-fresh/}
        \end{itemize}
        (See \url{http://www.libreoffice.org/get-help/install-howto/}
        for installation instructions)

        \noindent
        You can download the Spyder IDE ({\tt spyder-2.3.8}) from:
        \begin{itemize}
            \item Windows: Installed with {\tt Python(x,y)-2.7.9.0} above
            \item Mac: \url{ftp://ftp.ee.up.ac.za/pub/mac/python/spyder-2.3.8-py2.7.dmg}
            \item Vendor: \url{https://github.com/spyder-ide/spyder/releases}
        \end{itemize}

        \noindent
        \textbf{Note: The \url{ftp://ftp.ee.up.ac.za/pub/} server is only
        accessible on campus !!}
